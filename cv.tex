%!TEX TS-program = xelatex
\documentclass[print]{friggeri-cv}
%\documentclass[]{friggeri-cv}

\begin{document}
\header{elias}{obeid}
    {master of science in computer science and engineering}

% In the aside, each new line forces a line break
\begin{aside}
  \section{about}
    Elias Khazen Obeid
    February 7th, 1991
    ~
    Lake Shore Tower 1, Cluster Y, Jumeirah Lake Towers, Dubai
    United Arab Emirates
    %Grøntoften
    %2870 Dyssegård
    %Denmark
    ~
    %\textbf{mobile} +45 71 20 29 91
    +971 50 179 8766
    %\href{mailto:ekobeid@gmail.com}{ekobeid@gmail.com}
    \href{mailto:elias@seez.co}{elias@seez.co}
    \href{mailto:elias@munk.ai}{elias@munk.ai}
    %\href{http://obeid.dk}{http://obeid.dk}
    %\href{https://www.facebook.com/eliaskhazenobeid}{fb://eliasobeid}
    \href{https://github.com/obeyed}{github.com/obeyed}
    \href{https://www.linkedin.com/in/eliasobeid}{linkedin.com/eliasobeid}
  \section{language}
    trilingual danish/english/arabic
  \section{programming and markup}
    C, C++, C\#, CUDA, Java, JavaScript, Python, Ruby, Lua, 
    SQL, NoSQL, ReQL, 
    Haskell (learning),
    Git, SVN, \LaTeX{}, Unix, Windows
  \section{main interests}
    cryptology
    machine learning
    computer security
    multi-agent system
    artificial intelligence
    software engineering
    software development
\end{aside}

\section{education}

\begin{entrylist}
  \entry
    {2014--2016}
    {M.Sc. {\normalfont Computer Science and Engineering}}
    {The Technical University of Denmark}
    {Four semesters (120 ECTS) of four months each.
	Master thesis with the topic of machine translation and natural language processing.}
  \entry
    {2011--2014}
    {B.Sc. {\normalfont Computer Science}}
    {The University of Aalborg}
    {Six semesters (180 ECTS) of four months each.
	Bachelor project with the topic of genetic algorithms.}
  \entry
    {2007--2010}
    {Student}
    {The Technical Gymnasium of Aalborg}
    {Communication and Media}
%  \entry
%    {1998–2007}
%    {State school pupil}
%    {Mellervangskolen}
%    {0th to 9th grade}
\end{entrylist}

\section{work experience}

\begin{entrylist}
  \entry
    {since 2016}
    {Backend architect}
    {Seez}
    {Main fields of responsibility are the backend and API architecture, data management, and data crawling. 
	Furthermore, I partake in the development of artificially intelligent autonomous systems.}
  \entry
    {since 2016}
    {Co-founder \& machine learning consultant}
    {munk.ai}
    {Startup company (in collaboration with co-authors of master thesis) which will provide API's for image recognition with a mission of providing generalized image recognition.}
  \entry
    {2015--2016}
    {Student developer}
    {Berlingske People A/S, Radio24syv}
    {Development of website, API infrastructure, and other IT tools used internally. Mainly development with NodeJS with ExpressJs, AngularJS, and C\#.}
  \entry
    {2014--2015}
    {Student developer}
    {Falck Healthcare A/S}
    {Development of a booking system for a service, that Falck Healthcare offers its clients. Mainly web development with Ruby on Rails.}
  \entry
    {08–12 2013}
    {Assistant teacher}
    {The University of Aalborg}
    {Assistant teacher in imperative programming with C.}
%  \entry
%    {2012--2014}
%    {Ambassador}
%    {The University of Aalborg}
%    {Ambassador for Computer Science}
%  \entry
%    {08–11 2011}
%    {Salesmen}
%    {LN Eurocom}
%    {Sale of news paper over the phone for Nordjyske Stiftstidende}
  \entry
    {02–07 2011}
    {Private (national service)}
    {Danish Emergency Management Agency}
    {Firefighting education and emergency duty.}
%  \entry
%    {2010–2011}
%    {Shop assistant}
%    {Twenty4-7, Aalborg}
%    {Customer service, cleaning, replenishment of stock}
%  \entry
%    {2008–2010}
%    {Shop assistant/driver}
%    {La Pronto Pizza, Aalborg}
%    {Customer service, cleaning, cooking and delivery}
\end{entrylist}

\section{publications}

\begin{entrylist}
  \entry
    {Oct 2016}
    {Neural Machine Translation with Characters and Hierarchical Encoding\\}
    {\href{https://arxiv.org/abs/1610.06550}{arxiv.org/abs/1610.06550}}
    {We propose a model for translating between languages with a hierarchical architecture. It is fed with sequences of 
    characters from a source language and produces a translated sequence of characters in a target language.
    We argue that, by comparison to current state of the art models, this hierarchical architecture reduces computational 
    complexity, and that it improves translation performance.}
\end{entrylist}

\section{other activity}

\begin{entrylist}
  \entry
    {2011--2011}
    {Education in Firefighting}
    {Danish Emergency Management Agency}
    {Fire, Rescue, Environment and Communication.}
  \entry
    {2013--2014}
    {Member of the Board of Studies}
    {The University of Aalborg}
    {Discussions of applications of qualifications and exemptions, study lines, quality assurance of education and the like.}
%  \entry
%  {2013}
%  {Set up of website with WordPress}
%  {Emergency Center of Aalborg}
%  {Volunteer work.
%  \textit{The website is no longer in use.}
%  Set up website for the volunteers associated with The Emergency Center of Aalborg.
%  The website was for authorised users where they could sign in, view activity calendar, view images and relevant documents, write stories, and find other relevant information.
%  }
%  \entry
%  {2012 \& 2013}
%  {Rus-instructor}
%  {The University of Aalborg}
%  {Intruktor for new students}
  \entry
    {2012--2014}
    {Volunteer mentor}
    {Danish Red Cross Youth}
    {A project called \href{http://www.urk.dk/solskinsunge/}{Sunshine youth}, with the purpose of motivating youngsters for an education in Aalborg (Denmark).}
%  \entry
%  {2011-2014}
%  {Volunteer firefighter}
%  {Emergency Center of Aalborg}
%  {Maintenance of my firefighting education}
%  \entry
%  {since 2009}
%  {Driver's license}
%  {}{}
\end{entrylist}

\end{document}
