%!TEX TS-program = xelatex
%\documentclass[print]{friggeri-cv}
\documentclass[]{friggeri-cv}

\begin{document}
\header{elias}{obeid}
    {master of science in computer science and engineering}

% In the aside, each new line forces a line break
\begin{aside}
  \section{about}
    Elias Khazen Obeid
    February, 1991
    Danish citizen
    ~
    Copenhagen, Denmark
    %Grøntoften
    %2870 Dyssegård
    %Denmark
    ~
    %\textbf{mobile} 
    +45 7120 2991
    %\href{mailto:ekobeid@gmail.com}{ekobeid@gmail.com}
    %\href{mailto:elias@obeid.dk}{elias@obeid.dk}
    %\href{mailto:elias@seez.co}{elias@seez.co}
    \href{mailto:elias@munk.ai}{elias@munk.ai}
    %\href{http://obeid.dk}{http://obeid.dk}
    %\href{https://www.facebook.com/eliaskhazenobeid}{fb://eliasobeid}
    \href{https://github.com/obeyed}{github.com/obeyed}
    \href{https://www.linkedin.com/in/eliasobeid}{linkedin.com/eliasobeid}
  \section{language}
    trilingual danish/english/arabic
  \section{programming, markup, and tools}
  	Python, PyTorch, CUDA, C, C++, C\#,
    Java, JavaScript, Ruby, Haskell~(learning),
    SQL, NoSQL, ReQL,
    Git, SVN, Docker, Vim, Terminal, \LaTeX{}, Linux, Windows
  \section{main interests}
    deep learning
    computer security
    software engineering
\end{aside}

\section{education}

\begin{entrylist}
  \entry
    {2014--2016}
    {M.Sc. {\normalfont Computer Science and Engineering}}
    {The Technical University of Denmark}
    {Four semesters (120 ECTS) of four months each.
	Master thesis with the topic of machine translation and natural language processing.}
  \entry
    {2011--2014}
    {B.Sc. {\normalfont Computer Science}}
    {The University of Aalborg}
    {Six semesters (180 ECTS) of four months each.
	Bachelor project with the topic of genetic algorithms.}
%  \entry
%    {2007--2010}
%    {Student}
%    {The Technical Gymnasium of Aalborg}
%    {Communication and Media}
%  \entry
%    {1998–2007}
%    {State school pupil}
%    {Mellervangskolen}
%    {0th to 9th grade}
\end{entrylist}

\section{work experience}

\begin{entrylist}
  \entry
    {since 2018}
    {Co-founder \& deep learning consultant}
    {\href{http://www.munk.ai/}{munk.ai}}
    {
      Ambition to create leading research in natural language processing in Denmark.
      Our hope is to bridge the gap between research and product, such that state of the art deep learning technologies are no longer Silicon Valley proprietary assets.
    }
  \entry
    {2016--2018}
    {Systems architect}
    {Seez}
    {
      Built scalable and secure back-end to handle thousands of users on mobile apps.
      Server maintenance, file storage, and DNS hosted by AWS.
      Data crawling, data handling, and data processing.
      Applied deep learning techniques to approximate prices of the used car market.
    
      The founder has a vision of making it easier and more transparent to sell/buy a used car.
      First round of investment on \$1.8M with an evaluation of \$6M.    
	  }
  \entry
    {2015--2016}
    {Student developer}
    {Berlingske People A/S, Radio24syv}
    {
      Website, API infrastructure, and other IT tools used internally. 
      NodeJS with ExpressJs, AngularJS, and C\# with Dokku (local hosting) and MongoDB for storage.
      Hosted with Heroku and custom local servers.
    }
  \entry
    {2014--2015}
    {Student developer}
    {Falck Healthcare A/S}
    {
      Web development with Ruby on Rails using a PostgreSQL database.
    }
  \entry
    {08–12 2013}
    {Assistant teacher}
    {The University of Aalborg}
    {Assistant teacher in imperative programming with C.}
%  \entry
%    {2012--2014}
%    {Ambassador}
%    {The University of Aalborg}
%    {Ambassador for Computer Science}
%  \entry
%    {08–11 2011}
%    {Salesmen}
%    {LN Eurocom}
%    {Sale of news paper over the phone for Nordjyske Stiftstidende}
  \entry
    {02–07 2011}
    {Private (national service)}
    {Danish Emergency Management Agency}
    {Firefighting education and emergency duty.}
%  \entry
%    {2010–2011}
%    {Shop assistant}
%    {Twenty4-7, Aalborg}
%    {Customer service, cleaning, replenishment of stock}
%  \entry
%    {2008–2010}
%    {Shop assistant/driver}
%    {La Pronto Pizza, Aalborg}
%    {Customer service, cleaning, cooking and delivery}
\end{entrylist}

\section{publications}

\begin{entrylist}
  \entry
    {Oct 2016}
    {Neural Machine Translation with Characters and Hierarchical Encoding\\}
    {\href{https://arxiv.org/abs/1610.06550}{arxiv.org/abs/1610.06550}}
    {In NIPS RNN Symposium 2016.}
\end{entrylist}

%\section{other activity}

%\begin{entrylist}
%  \entry
%    {2011--2011}
%    {Education in Firefighting}
%    {Danish Emergency Management Agency}
%    {Fire, Rescue, Environment and Communication.}
%  \entry
%    {2013--2014}
%    {Member of the Board of Studies}
%    {The University of Aalborg}
%    {Discussions of applications of qualifications and exemptions, study lines, quality assurance of education and the like.}
%  \entry
%  {2013}
%  {Set up of website with WordPress}
%  {Emergency Center of Aalborg}
%  {Volunteer work.
%  \textit{The website is no longer in use.}
%  Set up website for the volunteers associated with The Emergency Center of Aalborg.
%  The website was for authorised users where they could sign in, view activity calendar, view images and relevant documents, write stories, and find other relevant information.
%  }
%  \entry
%  {2012 \& 2013}
%  {Rus-instructor}
%  {The University of Aalborg}
%  {Intruktor for new students}
%  \entry
%    {2012--2014}
%    {Volunteer mentor}
%    {Danish Red Cross Youth}
%    {A project called \href{http://www.urk.dk/solskinsunge/}{Sunshine youth}, with the purpose of motivating youngsters for an education in Aalborg (Denmark).}
%  \entry
%  {2011-2014}
%  {Volunteer firefighter}
%  {Emergency Center of Aalborg}
%  {Maintenance of my firefighting education}
%  \entry
%  {since 2009}
%  {Driver's license}
%  {}{}
%\end{entrylist}

\end{document}
