%!TEX TS-program = xelatex
%\documentclass[print]{friggeri-cv}
\documentclass[]{friggeri-cv}

\begin{document}
\header{elias}{obeid}
    {postgraduate in computer science and engineering}

% In the aside, each new line forces a line break
\begin{aside}
  \section{about}
    Elias Khazen Obeid
    February 7th, 1991
    Grøntoften 
    2870 Dyssegård
    Denmark
    ~
    \textbf{mobile} 71 20 29 91
    \href{mailto:ekobeid@gmail.com}{ekobeid@gmail.com}
    \href{http://obeid.dk}{http://obeid.dk}
    \href{http://facebook.com/eliaskhazenobeid}{fb://eliasobeid}
    \href{http://www.linkedin.com/in/eliasobeid}{in://eliasobeid}
  \section{language}
    trilingual danish/english/arabic
  \section{programming and markup}
    JavaScript, Java,
    Python, C, C++, C\#,
    Ruby on Rails,
    HTML, CSS and \LaTeX{}
\end{aside}

\section{education}

\begin{entrylist}
  \entry
    {since 2014}
    {Postgraduate {\normalfont in Computer Science and Engineering}\\}
    {The Technical University of Denmark}
    {Four semesters (120 ECTS) of four months each --- 13- and 3-week courses per semester}
  \entry
    {2011–2014}
    {B.Sc. {\normalfont in Computer Science}}
    {The University of Aalborg}
    {Six semesters (180 ECTS) of four months each --- one project and three courses per semester}
  \entry
    {2011–2011}
    {Education in Firefighting}
    {Danish Emergency Management Agency}
    {Fire, Rescue, Environment and Communication}
  \entry
    {2007–2010}
    {Student}
    {The Technical Gymnasium of Aalborg}
    {Communication and Media}
  \entry
    {1998–2007}
    {State school pupil}
    {Mellervangskolen}
    {0th to 9th grade}
\end{entrylist}

\section{work experience}

\begin{entrylist}
  \entry
    {since 2015}
    {Student developer}
    {Berlingske People A/S, Radio24syv}
    {Development of website and API infrastructure for website and mobile app. Mainly development with Javascript, AngularJS, and C\#}
  \entry
    {2014-2015}
    {Student developer}
    {Falck Healthcare A/S}
    {Development of a booking system for a service, that Falck Healthcare offers its clients. Mainly web development with Ruby on Rails}
  \entry
    {08–12 2013}
    {Assistant teacher}
    {The University of Aalborg}
    {Assitant teacher in imperative programming with C}
  \entry
    {2012–2014}
    {Ambassador}
    {The University of Aalborg}
    {Ambassador for Computer Science}
  \entry
    {08–11 2011}
    {Salesmen}
    {LN Eurocom}
    {Sale of news paper over the phone for Nordjyske Stiftstidende}
  \entry
    {02–07 2011}
    {Private (national service)}
    {Danish Emergency Management Agency}
    {Firefighting education and emergency duty}
  \entry
    {2010–2011}
    {Shop assistant}
    {Twenty4-7, Aalborg}
    {Customer service, cleaning, replenishment of stock}
  \entry
    {2008–2010}
    {Shop assistant/driver}
    {La Pronto Pizza, Aalborg}
    {Customer service, cleaning, cooking and delivery}
\end{entrylist}

\section{interests and spare time}
I have always been interested in technology. I started studying computer science to learn more about programming, computers and IT in general. Machine learning and issues of computer security have caught my attention, and these are subjects I am focusing on for my Masters degree.

I like to read and watch anything exiting about nature and the universe. I love the tv series ``Cosmos: A SpaceTime Odyssey'' with Neil deGrasse Tyson, I am reading the book ``A Short History of Nearly Everything'' by Bill Bryson and will continue with ``A Brief History of Time'' by Stephen Hawkings and ``Why Does E=mc2? (And Why Should We Care?)'' by Brian Cox afterwards.

\section{qualifications}

I am a quick learner and curious of nature. I can easily familiarize myself with new topics. I have great interest in programming. I have experience with content management systems like WordPress. Version control systems, such as Git and SVN, which I work with daily. Additionally, I have knowledge of operating systems, especially Windows and Ubuntu.

\section{other activity}

\begin{entrylist}
  \entry
  {2013–2014}
  {Member of the Board of Studies}
  {The University of Aalborg}
  {Discussions of applications of qualifications and exemptions, study lines, quality assurance of education and the like.}
  \entry
  {2013}
  {Set up of website with WordPress}
  {Emergency Center of Aalborg}
  {Volunteer work.
  \textit{The website is no longer in use.}
  Set up website for the volunteers associated with The Emergency Center of Aalborg.
  The website was for authorised users where they could sign in, view activity calendar, view images and relevant documents, write stories, and find other relevant information.
  }
  \entry
  {2012 \& 2013}
  {Rus-instructor}
  {The University of Aalborg}
  {Intruktor for new students}
  \entry
  {2012–2014}
  {Volunteer mentor}
  {Danish Red Cross Youth}
  {Part of a project \href{http://www.urk.dk/solskinsunge/}{Sunshine youth}, where we would arrange social events with the youth of a state school (Tornhøjskolen) in Eastern Aalborg.}
  \entry
  {2011-2014}
  {Volunteer firefighter}
  {Emergency Center of Aalborg}
  {Maintenance of my firefighting education}
  \entry
  {since 2009}
  {Driver's license}
  {}{}
\end{entrylist}

\end{document}
