%!TEX TS-program = xelatex
%\documentclass[print]{friggeri-cv}
\documentclass[]{friggeri-cv}

\begin{document}
\header{elias}{obeid}
    {master of science in computer science and engineering}

% In the aside, each new line forces a line break
\begin{aside}
  \section{about}
    Elias Khazen Obeid
    February, 1991
    Danish citizen
    ~
    Copenhagen, Denmark
    %Grøntoften
    %2870 Dyssegård
    %Denmark
    ~
    %\textbf{mobile} 
    +45 7120 2991
    %\href{mailto:ekobeid@gmail.com}{ekobeid@gmail.com}
    \href{mailto:elias@obeid.dk}{elias@obeid.dk}
    \href{mailto:elias@seez.co}{elias@seez.co}
    \href{http://obeid.dk}{http://obeid.dk}
    %\href{https://www.facebook.com/eliaskhazenobeid}{fb://eliasobeid}
    \href{https://github.com/obeyed}{github.com/obeyed}
    \href{https://www.linkedin.com/in/eliasobeid}{linkedin.com/eliasobeid}
  \section{language}
    trilingual danish/english/arabic
  \section{programming, markup, and tools}
  	Python, PyTorch, CUDA, C, C++, C\#,
    Java, JavaScript, Ruby, Haskell~(learning),
    SQL, NoSQL, ReQL,
    Git, SVN, Docker, Vim, Terminal, \LaTeX{}, Linux, Windows
  \section{main interests}
    deep learning
    computer security
    software engineering
\end{aside}

\section{education}

\begin{entrylist}
  \entry
    {2014--2016}
    {M.Sc. {\normalfont Computer Science and Engineering}}
    {The Technical University of Denmark}
    {Four semesters (120 ECTS) of four months each.
	Master thesis with the topic of machine translation and natural language processing.}
  \entry
    {2011--2014}
    {B.Sc. {\normalfont Computer Science}}
    {The University of Aalborg}
    {Six semesters (180 ECTS) of four months each.
	Bachelor project with the topic of genetic algorithms.}
\end{entrylist}

\section{work experience}

\begin{entrylist}
  \entry
    {Since 2016}
    {Systems architect}
    {Seez}
    {
      The founder had a vision of making it easier and more transparent to sell/buy a used car.
      For our first round of funding we raised \$1.8m.
      It has now evolved such that we among other services also provide cars for lease through the app.
    
      Initially I had the sole responsibility of building the backend, but almost three years down the road, we are now two senior engineers responsible for the still growing backend.
      We have systems scraping websites daily and a pipeline of cleaning that data such that it is ready to be served to the frontend through the API.
      Other than expanding the code base, responsibilities lie in maintaining the machines running the backend, and also providing my opinion on technical matters and participating in final decisions.
      
      Furthermore, I developed the first version of the company's web app, which is, of this writing, still in beta, but there are still a couple 100 weekly active users visiting the site. As of this writing, we have around 5k daily active users on our Android and iOS apps. All frontend devices communicate with the same API.
    
      To list a few services and tools, we use PostgreSQL, Python3, VueJS, Git and Docker. We run most of the backend with Docker Swarms. We use Scrapy and Selenium for web crawling. Using that data, we apply machine learning techniques for a large array of applications, which are use throughout the company.
	  }
  \entry
    {Since 2017}
    {Independent business consultant}
    {Bits by Obeid}
    {Providing general development on project basis and IT consultancy for other companies.}
  \entry
    {2015--2016}
    {Student developer}
    {Berlingske People A/S, Radio24syv}
    {
      Website, API infrastructure, and other IT tools used internally. 
      NodeJS with ExpressJs, AngularJS, and C\# with Dokku (local hosting) and MongoDB for storage.
      Hosted with Heroku and custom local servers.
    }
  \entry
    {2014--2015}
    {Student developer}
    {Falck Healthcare A/S}
    {
      Web development with Ruby on Rails using a PostgreSQL database.
    }
  \entry
    {08–12 2013}
    {Assistant teacher}
    {The University of Aalborg}
    {Assistant teacher in imperative programming with C.}
%  \entry
%    {2012--2014}
%    {Ambassador}
%    {The University of Aalborg}
%    {Ambassador for Computer Science}
%  \entry
%    {08–11 2011}
%    {Salesmen}
%    {LN Eurocom}
%    {Sale of news paper over the phone for Nordjyske Stiftstidende}
  \entry
    {02–07 2011}
    {Private (national service)}
    {Danish Emergency Management Agency}
    {Firefighting education and emergency duty.}
%  \entry
%    {2010–2011}
%    {Shop assistant}
%    {Twenty4-7, Aalborg}
%    {Customer service, cleaning, replenishment of stock}
%  \entry
%    {2008–2010}
%    {Shop assistant/driver}
%    {La Pronto Pizza, Aalborg}
%    {Customer service, cleaning, cooking and delivery}
\end{entrylist}

\newpage

\section{publications}

\begin{entrylist}
  \entry
    {Oct 2016}
    {Neural Machine Translation with Characters and Hierarchical Encoding\\}
    {\href{https://arxiv.org/abs/1610.06550}{arxiv.org/abs/1610.06550}}
    {In NIPS RNN Symposium 2016.}
\end{entrylist}

\section{other activity}

\begin{entrylist}
  \entry
    {2011--2011}
    {Education in Firefighting}
    {Danish Emergency Management Agency}
    {Fire, Rescue, Environment and Communication.}
  \entry
    {2013--2014}
    {Member of the Board of Studies}
    {The University of Aalborg}
    {Discussions of applications of qualifications and exemptions, study lines, quality assurance of education and the like.}
%  \entry
%  {2013}
%  {Set up of website with WordPress}
%  {Emergency Center of Aalborg}
%  {Volunteer work.
%  \textit{The website is no longer in use.}
%  Set up website for the volunteers associated with The Emergency Center of Aalborg.
%  The website was for authorised users where they could sign in, view activity calendar, view images and relevant documents, write stories, and find other relevant information.
%  }
%  \entry
%  {2012 \& 2013}
%  {Rus-instructor}
%  {The University of Aalborg}
%  {Intruktor for new students}
%  \entry
%    {2012--2014}
%    {Volunteer mentor}
%    {Danish Red Cross Youth}
%    {A project called \href{http://www.urk.dk/solskinsunge/}{Sunshine youth}, with the purpose of motivating youngsters for an education in Aalborg (Denmark).}
%  \entry
%  {2011-2014}
%  {Volunteer firefighter}
%  {Emergency Center of Aalborg}
%  {Maintenance of my firefighting education}
%  \entry
%  {since 2009}
%  {Driver's license}
%  {}{}
\end{entrylist}

\end{document}
